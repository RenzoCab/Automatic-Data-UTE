\PassOptionsToPackage{table}{xcolor}
\documentclass[aspectratio=169]{beamer}\usepackage[utf8]{inputenc}
\usepackage{lmodern}
\usepackage[english]{babel}
\usepackage{color}
\usepackage{amsmath,mathtools}
\usepackage{booktabs}
\usepackage{mathptmx}
\usepackage[11pt]{moresize}
\usepackage{hyperref}
\usepackage{commath}
\usepackage{bm}
\usepackage{subfigure}
\usepackage{siunitx}

\setbeamertemplate{navigation symbols}{}
\setbeamersize{text margin left=5mm,text margin right=5mm}
\setbeamertemplate{caption}[numbered]
\addtobeamertemplate{navigation symbols}{}{
\usebeamerfont{footline}
\usebeamercolor[fg]{footline}
\hspace{1em}
\insertframenumber/\inserttotalframenumber}

\newcommand{\R}{\mathbb{R}}
\newcommand{\E}{\mathbb{E}}
\newcommand{\N}{\mathbb{N}}
\newcommand{\Z}{\mathbb{Z}}
\newcommand{\V}{\mathbb{V}}
\newcommand{\Q}{\mathbb{Q}}
\newcommand{\K}{\mathbb{K}}
\newcommand{\C}{\mathbb{C}}
\newcommand{\T}{\mathbb{T}}
\newcommand{\I}{\mathbb{I}}

\title{Wind Power Data for the Project}
\subtitle{Renzo Miguel Caballero Rosas}

\begin{document}

\begin{frame}
\titlepage
\end{frame}

\setbeamercolor{background canvas}{bg=white!10}
\begin{frame}\frametitle{About the real production (ADME Vs. UTE):}

We have two sources that provide real wind power production. ADME provides with a frequency of six measurements per hour and UTE with only one measurement per hour. However, there exists a delay between the sources of approximately 100 minutes; as the forecast matches with the real production of UTE, we decide to translate in time the data from ADME, so the three paths match in time.\\
\quad\\
There exists a second negative effect: It seems that ADME shows when the wind power is artificially truncated, and UTE does not. To find the days where this effect is present, we do a statistical study of the data.\\
\quad\\
We define $Y(t)$ and $X(t)$ real production for the same day of ADME and UTE, respectively. Also, we define the error $Z(t)=X(t)-Y(t)$ for some appropriate $t$. 

\end{frame}

\setbeamercolor{background canvas}{bg=white!10}
\begin{frame}\frametitle{About the real production (ADME Vs. UTE):}

We define $N=145$ the number of measurement per path, $M=365$ the amount of paths, $P_{max}=\SI{1474}{\mega\watt}$ the installed wind power during all 2019, and $T=\SI{24}{\hour}$ the time normalization constant.\\
\quad\\
Then, we want to plot an histogram for the set $\{\hat{Z^i}\}_{i=1}^M$, where
\begin{equation}
\hat{Z^i}=\frac{1}{N\cdot P_{max}}\sum_{j=0}^{N-1}Z^i\left(T\frac{j}{N-1}\right)\quad\text{for all}\quad i\in\{1,\dots,M\}.
\label{1}
\end{equation}
Notice that (\ref{1}) is an average of the relative error between UTE and ADME. We do a linear interpolation to fill UTE's paths.

\end{frame}

\begin{frame}\frametitle{Histogram for all the 365 days:}
\begin{figure}[ht!]
\centering
{\includegraphics[width=0.5\columnwidth]{all2019.eps}}
\end{figure}
\alert{Notice all the positive relative error. To be positive implies that UTE shows a greater value, which has sense since ADME is showing the limitations.}
\end{frame}

\begin{frame}\frametitle{Histogram for 297 days:}
\begin{figure}[ht!]
\centering
{\includegraphics[width=0.5\columnwidth]{partially2019.eps}}
\end{figure}
\alert{We remove the days where the data is all zeros (2 days in total), and when the mean relative error is $\geq0.007$ or $\leq-0.013$ (108 days in total).}
\end{frame}

\end{document}