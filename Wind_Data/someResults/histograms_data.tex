\PassOptionsToPackage{table}{xcolor}
\documentclass[aspectratio=169]{beamer}\usepackage[utf8]{inputenc}
\usepackage{lmodern}
\usepackage[english]{babel}
\usepackage{color}
\usepackage{amsmath,mathtools}
\usepackage{booktabs}
\usepackage{mathptmx}
\usepackage[11pt]{moresize}
\usepackage{hyperref}
\usepackage{commath}
\usepackage{bm}
\usepackage{subfigure}
\usepackage{siunitx}

\setbeamertemplate{navigation symbols}{}
\setbeamersize{text margin left=5mm,text margin right=5mm}
\setbeamertemplate{caption}[numbered]
\addtobeamertemplate{navigation symbols}{}{
\usebeamerfont{footline}
\usebeamercolor[fg]{footline}
\hspace{1em}
\insertframenumber/\inserttotalframenumber}

\newcommand{\R}{\mathbb{R}}
\newcommand{\E}{\mathbb{E}}
\newcommand{\N}{\mathbb{N}}
\newcommand{\Z}{\mathbb{Z}}
\newcommand{\V}{\mathbb{V}}
\newcommand{\Q}{\mathbb{Q}}
\newcommand{\K}{\mathbb{K}}
\newcommand{\C}{\mathbb{C}}
\newcommand{\T}{\mathbb{T}}
\newcommand{\I}{\mathbb{I}}

\title{Wind Power Data:\\
Histograms for different transformations}
\subtitle{Renzo Miguel Caballero Rosas}

\begin{document}

\begin{frame}
\titlepage
\end{frame}

\begin{frame}\frametitle{Data points:}
\begin{figure}[ht!]
\centering
{\includegraphics[width=0.32\columnwidth]{./forPaper/Gauss_Approx_Measurements_Error.eps}}
{\includegraphics[width=0.32\columnwidth]{./forPaper/Gauss_Approx_Measurements_Lamperti.eps}}
{\includegraphics[width=0.32\columnwidth]{./forPaper/Gauss_Approx_Measurements_NT.eps}}
\end{figure}
Here we can see the histograms for the data points. {\color{blue}In blue}, the error measutrements, corresponding to $V(t_i)=X(t_i)-p(t_i)$. {\color{orange}In orange}, the Lamperti transform $\psi(V(t_i))$. Finally, {\color{green}in green}, the A\"it-Sahilia's transform.
\end{frame}

\begin{frame}\frametitle{Data transitions:}
\begin{figure}[ht!]
\centering
{\includegraphics[width=0.32\columnwidth]{./forPaper/Gauss_Approx_Err.eps}}
{\includegraphics[width=0.32\columnwidth]{./forPaper/Gauss_Approx_Lam.eps}}
{\includegraphics[width=0.32\columnwidth]{./forPaper/Gauss_Approx_NT.eps}}
\end{figure}
Here we see the transitions' histograms for each process described in the previous slide.
\end{frame}

\end{document}