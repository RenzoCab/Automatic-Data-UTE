\let\mypdfximage\pdfximage
\def\pdfximage{\immediate\mypdfximage}

\documentclass[12pt]{article}
\usepackage[table]{xcolor}
\usepackage[margin=1in]{geometry} 
\usepackage{amsmath,amsthm,amssymb}
\usepackage[english]{babel}
\usepackage{graphicx}
\usepackage{tcolorbox}
\usepackage{enumitem}
\usepackage{hyperref}
\usepackage{fmtcount}
\usepackage{listings}
\usepackage{blkarray}
\usepackage{float}
\usepackage{bm}
\usepackage{subfigure}
\usepackage{booktabs}
\usepackage[maxfloats=256]{morefloats}
\usepackage{siunitx}
\usepackage{forloop}
%\usepackage[space]{grffile}

% To do computations with the counters:
\usepackage{calc}

\usepackage{geometry}
\geometry{a4paper,
          left=            5mm,
          right=           5mm,
          top=             15mm,
          bottom=          25mm,
          heightrounded}

\maxdeadcycles=10000

\setcounter{secnumdepth}{5}
\setcounter{tocdepth}{5}

\newtheorem{theorem}{Theorem}[section]
\newtheorem{corollary}{Corollary}[theorem]
\newtheorem{lemma}[theorem]{Lemma}
\newtheorem{proposition}[theorem]{proposition}
\newtheorem{exmp}{Example}[section]\newtheorem{definition}{Definition}[section]
\newtheorem{remark}{Remark}
\newtheorem{ex}{Exercise}
\theoremstyle{definition}
\theoremstyle{remark}
\bibliographystyle{elsarticle-num}

\DeclareMathOperator{\sinc}{sinc}
\newcommand{\RNum}[1]{\uppercase\expandafter{\romannumeral #1\relax}}
\newcommand{\N}{\mathbb{N}}
\newcommand{\Z}{\mathbb{Z}}
\newcommand{\R}{\mathbb{R}}
\newcommand{\E}{\mathbb{E}}
\newcommand{\matindex}[1]{\mbox{\scriptsize#1}}
\newcommand{\V}{\mathbb{V}}
\newcommand{\Q}{\mathbb{Q}}
\newcommand{\K}{\mathbb{K}}
\newcommand{\C}{\mathbb{C}}
\newcommand{\prob}{\mathbb{P}}
\newcommand{\Date}{This text will change :D}
\newcommand{\DateDay}{This text will change :(}
\newcommand{\DateMonth}{This text will change :)}
\newcommand{\mysize}{0.28}
\newcommand{\mysizeTwo}{0.20}


\lstset{numbers=left, numberstyle=\tiny, stepnumber=1, numbersep=5pt}

\begin{document}
\title{Real Production and Forecast (MTLOG 0100)\\
from 01:00 to 01:00 hrs (next day).}
%\author{Renzo Miguel Caballero Rosas} 
\maketitle

\begin{figure}[ht!]
\centering
{\includegraphics[width=0.48\columnwidth]{1.eps}}\quad
{\includegraphics[width=0.48\columnwidth]{2.eps}}\\
\quad\\
\quad\\
{\includegraphics[width=0.48\columnwidth]{3.eps}}\quad
{\includegraphics[width=0.48\columnwidth]{4.eps}}
\end{figure}

\newcounter{i}
\newcounter{j}

\forloop{i}{0}{\value{i} < 60}{

%\setcounter{j}{7*(\value{i}-1)+1}


\begin{figure}[ht!]
\centering
\setcounter{j}{\value{i}*6+5}
{\includegraphics[width=0.48\columnwidth]{\arabic{j}.eps}}\quad
\setcounter{j}{\value{i}*6+6}
{\includegraphics[width=0.48\columnwidth]{\arabic{j}.eps}}\\
\quad\\
\quad\\
\setcounter{j}{\value{i}*6+7}
{\includegraphics[width=0.48\columnwidth]{\arabic{j}.eps}}\quad
\setcounter{j}{\value{i}*6+8}
{\includegraphics[width=0.48\columnwidth]{\arabic{j}.eps}}\\
\quad\\
\quad\\
\setcounter{j}{\value{i}*6+9}
{\includegraphics[width=0.48\columnwidth]{\arabic{j}.eps}}\quad
\setcounter{j}{\value{i}*6+10}
{\includegraphics[width=0.48\columnwidth]{\arabic{j}.eps}}
\end{figure}

}

\end{document}